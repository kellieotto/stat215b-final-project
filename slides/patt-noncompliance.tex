\documentclass{beamer}
%%%%%%%%%%%%%%%%%%%%%%%%%%%%%%%%%%%%%%%%%%%%%%%%%%%%%%%%%%%%%%%

% Define packages
\usepackage{beamerthemesplit}
\usepackage{graphicx,amsfonts,psfrag,layout,subcaption,array,longtable,lscape,booktabs,dcolumn,natbib,amsmath,amssymb,amssymb,amsthm,setspace,epigraph,chronology,color, colortbl,caption}
\usepackage[]{graphicx}\usepackage[]{color}
\usepackage[page]{appendix}
\usepackage{hyperref, url} %For submission, uncheck and fix URLs ($$)
\usepackage[section]{placeins}
\usepackage[linewidth=1pt]{mdframed}

% Footnotes stick at the bottom
\usepackage[bottom]{footmisc}

% New footnote characters
\usepackage{footmisc}
\DefineFNsymbols{mySymbols}{{\ensuremath\dagger}{\ensuremath\ddagger}\S\P
   *{**}{\ensuremath{\dagger\dagger}}{\ensuremath{\ddagger\ddagger}}}
\setfnsymbol{mySymbols}

% New tabular environment
\usepackage{tabularx}
\newcolumntype{Y}{>{\raggedleft\arraybackslash}X}% raggedleft column X

% Define appendix 
\renewcommand*\appendixpagename{Appendix}
\renewcommand*\appendixtocname{Appendix}

% Position floats
\renewcommand{\textfraction}{0.05}
\renewcommand{\topfraction}{0.95}
\renewcommand{\bottomfraction}{0.95}
\renewcommand{\floatpagefraction}{0.35}
\setcounter{totalnumber}{5}

% Colors for highlighting tables
\definecolor{Gray}{gray}{0.9}

% Different font in captions
\newcommand{\captionfonts}{\scriptsize}

\makeatletter  % Allow the use of @ in command names
\long\def\@makecaption#1#2{%
  \vskip\abovecaptionskip
  \sbox\@tempboxa{{\captionfonts #1: #2}}%
  \ifdim \wd\@tempboxa >\hsize
    {\captionfonts #1: #2\par}
  \else
    \hbox to\hsize{\hfil\box\@tempboxa\hfil}%
  \fi
  \vskip\belowcaptionskip}
%\makeatother   % Cancel the effect of \makeatletter
 
% Number assumptions
\newtheorem*{assumption*}{\assumptionnumber}
\providecommand{\assumptionnumber}{}
\makeatletter
\newenvironment{assumption}[2]
 {%
  \renewcommand{\assumptionnumber}{Assumption #1}%
  \begin{assumption*}%
  \protected@edef\@currentlabel{#1}%
 }
 {%
  \end{assumption*}
 }
\makeatother

% Macros
\newcommand{\Adv}{{\mathbf{Adv}}}       
\newcommand{\prp}{{\mathrm{prp}}}                  % How to define new commands 
\newcommand{\calK}{{\cal K}}
\newcommand{\outputs}{{\Rightarrow}}                
\newcommand{\getsr}{{\:\stackrel{{\scriptscriptstyle\hspace{0.2em}\$}}{\leftarrow}\:}}
\newcommand{\andthen}{{\::\;\;}}    %  \: \; for thinspace, medspace, thickspace
\newcommand{\Rand}[1]{{\mathrm{Rand}[{#1}]}}       % A command with one argument
\newcommand{\Perm}[1]{{\mathrm{Perm}[{#1}]}}       
\newcommand{\Randd}[2]{{\mathrm{Rand}[{#1},{#2}]}} % and with two arguments
\newcommand{\E}{\mathrm{E}}
\newcommand{\Var}{\mathrm{Var}}
\newcommand{\Cov}{\mathrm{Cov}}
\DeclareMathOperator*{\plim}{plim}
\newcommand{\ind}{\mathbb{I}} % Indicator function
\newcommand{\pr}{\mathbb{P}} % Generic probability
\newcommand{\ex}{\mathbb{E}} % Generic expectation
\newcommand{\cov}{\mathrm{Cov}}
\newcommand\independent{\protect\mathpalette{\protect\independenT}{\perp}}
\def\independenT#1#2{\mathrel{\rlap{$#1#2$}\mkern2mu{#1#2}}}
\newcommand{\possessivecite}[1]{\citeauthor{#1}'s [\citeyear{#1}]} 

% Making a DAG
\usepackage{tkz-graph}  
\usetikzlibrary{shapes.geometric}
\tikzstyle{VertexStyle} = [shape            = ellipse,
                               minimum width    = 6ex,%
                               draw]
 \tikzstyle{EdgeStyle}   = [->,>=stealth']      


%color
\usecolortheme[RGB={50,200,50}]{structure} 

\usetheme[secheader]{Boadilla} 
\setbeamertemplate{items}[default] 
\setbeamercovered{transparent}
\setbeamertemplate{blocks}[rounded][shadow=true] 
\setbeamertemplate{navigation symbols}{} 
\mode<presentation>
\title[]{title}

\author[J. Poulos, K. Ottoboni]{Jason Poulos, Kellie Ottoboni}
%\institute[UCB]{Travers Dept. of Political Science \\
%University of California, Berkeley}
\date[04/30/15]{}
\begin{document}

\frame{\titlepage}

\section[Introduction]{}

\begin{frame}
\frametitle{Motivation}
\begin{itemize}
\item 
\end{itemize}
\end{frame}

\begin{frame}
\frametitle{Overview of experiment}
\begin{itemize}
\item Oregon Health Insurance Experiment
\end{itemize}
\end{frame}

\section[Estimation]{}

\begin{frame}
\frametitle{Estimating treatment effects}
\begin{itemize}
\item Neyman-Rubin framework: each $i = \left\{1, ..., N \right\}$ participants have four potential outcomes, $Y_{ist}$ for $s = 0,1$ and $t = 0,1$
\item Define W, S, T, C, D, Y
\end{itemize}
\end{frame}


\begin{frame}
\frametitle{Estimating treatment effects}
\begin{figure}[h]
\centering
\begin{tikzpicture}[scale=1.5] 
\SetGraphUnit{2} 
\Vertex{W}  \NOEA(W){S} \EA(S){T}  \EA(W){C} \SOEA(T){D} \SOEA(C){Y} \SOWE(D){Y}
\Edges(W, S, T) \Edges(W,T) \Edges(W, C) \Edges(T, D) \Edges(C, D) \Edges(W, Y) \Edges(D, Y)
\end{tikzpicture}
\caption{Causal diagram indicating the conditional independence assumptions needed to estimate the PATT.}\label{fig:DAG}
\end{figure}

\end{frame}

\begin{frame}
\frametitle{Estimating treatment effects (cont.)}
\begin{assumption}{1}{}\label{consistency}
Consistency under parallel studies: for all $i$ and for $t=0, 1$,
$$Y_{i0t} = Y_{i1t}$$
\end{assumption}
\end{frame}

\begin{frame}
\frametitle{Estimating treatment effects (cont.)}
\begin{assumption}{2}{}\label{si_treat}
Strong ignorability of sample assignment for treated:
\begin{equation*}
(Y_{01}, Y_{11}) \independent S \mid (W, T=1,C = 1), 0 < \pr(S=1 \mid W, T=1,C = 1) <1 
\end{equation*}
\end{assumption}
\noindent Potential outcomes for treatment are independent of sample assignment for individuals with the same covariates $W$ and assignment to treatment.

\begin{assumption}{3}{}\label{si_ctrl}
Strong ignorability of sample assignment for controls:
\begin{equation*}
(Y_{00}, Y_{10}) \independent S \mid (W, T=1,C = 1), 0 < \pr(S=1 \mid W, T=1,C = 1) <1 
\end{equation*}\end{assumption}

\noindent Potential outcomes for control are independent of sample assignment for individuals with the same covariates $W$ and assignment to treatment.
\end{frame}

\begin{frame}
\frametitle{Estimating treatment effects (cont.)}
\begin{assumption}{4}{}\label{sutva}
Stable unit treatment value assumption (SUTVA):
\begin{equation*}
Y_{ist}^{L_i} = Y_{ist}^{L_j},  \forall i \neq j
\end{equation*}
where $L_j$ is the treatment and sample assignment vector for unit $j$. \end{assumption}
 
\begin{assumption}{5}{}\label{compl}
Conditional independence of compliance and assignment:
\begin{equation*}
C \independent T=1 \mid W, 0 < \pr(C = 1 \mid W) < 1
\end{equation*}
\end{assumption}
\end{frame}

\begin{frame}
\frametitle{Estimating treatment effects (cont.)}
\begin{assumption}{6}{}\label{monotonicity}
Monotonicity: 
\begin{equation*}
T_i \geq D_i, \forall i
\end{equation*}
\end{assumption}
\noindent This assumption implies that there are no defiers and that crossover is only possible from treatment to control.
\begin{assumption}{7}{}\label{ER}
Exclusion restriction: For non-compliers
\begin{equation*}
Y_{11} = Y_{10}
\end{equation*}  
\end{assumption}
\noindent The treatment assignment affects the response only through the treatment received.  In particular, the treatment effect may only be non-zero for compliers.  
\end{frame}



\begin{frame}
\frametitle{Estimating treatment effects (cont.)}
\begin{theorem}\label{thm1}
Under assumptions \eqref{consistency} - \eqref{ER},

$$\tau_{\text{PATT}} = \ex_{01}\left[  \ex\left(Y_{11} \mid S=1, D=1, W\right)\right]
-\ex_{01}\left[  \ex\left(Y_{10} \mid S=1, T=0, C=1, W\right) \right] $$

where $\ex_{01}\left[\ex(\cdot \mid\dots, W)\right]$ denotes the expectation with respect to the distribution of $W$ in the treated individuals in the target population.  
\end{theorem}
\end{frame}

\begin{frame} % This slide is too wordy
\frametitle{Estimation Procedure}
\begin{enumerate}
\item Using the group assigned to treatment in the RCT $(S=1, T=1)$, train a model to predict complier status $C$ using the covariates $W$.
\item Predict who in the RCT assigned to control \textit{would have} complied to treatment had they been assigned to the treatment group.
\item For the group of observed compliers to treatment and predicted compliers in the control group, train a model to predict the response using as features the covariates $W$ and the treatment $T$ (assigned and observed are the same, for these individuals).  This model gives $\ex(Y_{1t} \mid S=1, C=1, T=t, W)$ for $t = 0,1$.
\item For all individuals who received treatment in the population $(S=0, D=1)$, estimate their potential outcomes $Y_{10}$ and $Y_{11}$ using the model from step 3.  The mean counterfactual $Y_{11}$ minus the mean counterfactual $Y_{10}$ is the estimate of $\tau_{\text{PATT}}$.
\end{enumerate}
\end{frame}


%\section[Georgia land lotteries]{}
%
%\begin{frame}
%\frametitle{Lottery process}
%\begin{itemize}
%\item 1805 lottery created three new counties from Creek lands: Baldwin, Wayne, and Wilkinson; 1807 lottery extended boundaries of Baldwin and Wilkinson counties
%\item Land divided into districts and square lots of 202.5 acres each (490 acres for Wayne county)
%\item Prize tickets representing each lot placed in ``lottery wheel''
%\begin{itemize}
%\item Blank tickets equal in number to \# draws - \# prizes also placed in wheel
%\end{itemize}
%\item Eligibility extended to free white men 21+ (1 draw); orphaned children (1 draw); married men with children (2 draws) and widows with children (2 draws)
%\begin{itemize}
%\item 1807 rules: orphan families with both parents deceased (2 draws); widows (1 draw); free white unmarried females 21+ (1 draw); 1805 fortunate drawers excluded
%\item Entry fee: 12.5 cents per draw
%\end{itemize}
%\end{itemize}
%\end{frame}
%
%%County map
%\begin{frame}
%\begin{figure}[htbp] 
%   \centering
%   \includegraphics[width=4in]{county-map.pdf} 
%   \caption{Map of Georgia with 1807 county boundaries \citep{long1995}. The northernmost shaded counties are Baldwin and Wilkinson, respectively, and Wayne is the southernmost shaded county.}
%   \label{map}
%\end{figure}
%\end{frame}
%
\section[Simulation]{}

\begin{frame}
\frametitle{Simulation}
\begin{itemize}
\item describe simulation design?
\end{itemize}
\end{frame}

%Simulation results plot
\begin{frame}
\begin{figure}[htbp]
%\caption{Balance in treatment assignment for 1805 sample.}
\centering
   \includegraphics[width=\linewidth]{../paper/mse_boxplots_B5.pdf} 
 %  \footnotesize{Note: $p$ values are calculated using a two--sided randomization test ($\mathcal{L}=1,000$ iterations) for weighted difference of means between treatment and control groups. Refer to footnotes in Table \ref{balance-nom} and Figure \ref{qq} for variable descriptions.}
\label{simulation-plot}
\end{figure}
\end{frame}



\section[Data]{}

\begin{frame}
\frametitle{Data}
\begin{itemize}
\item List of 1805 lottery participants compiled by \citet{graham2005}
\item Fortunate drawer records for 1805 and 1807 lotteries \citep{graham2004,graham2011}
\item Roster of officeholders published by Georgia Archives (Trustee period -- 1847)
\item Roll call votes extracted from Journals of the House and Senate of the State of Georgia
\begin{itemize}
\item  15 votes: emancipate certain slaves; facilitate introduction of slaves into state and prevent slaves being carried out of the state; and punish slaves and free blacks
\end{itemize}
\item Individual property tax records (1790 -- 1865)  \citep{archives1890,blair1926}
\end{itemize}
\end{frame}

\begin{frame}
\frametitle{Linking participants with officeholders}
\begin{enumerate}
\item Manually deduplicate 1807 records matched with officeholders based on exact match of surname and Soundex codes of first name
\item Randomly split matched records into training (60\%) and test (40\%) sets
\item  Fit an algorithmic model using random forests \citep{breiman2001} on training set with features common to both datasets (test set error rate of 35\%)
\item Use model to deduplicate 1805 lottery records matched with officeholders
\end{enumerate}
\end{frame}




\section[Results]{}

%Treatment effect on officeholding
\begin{frame}
\frametitle{Results}
\begin{table}[htb]
\caption{Officeholding by treatment assignment.}   \label{lotteries} 
  \begin{tabularx}{\linewidth}{l*{8}{Y}}
    \toprule
    \multicolumn{8}{l}{\textbf{Panel A: 1805 sample}} \\
    \midrule
\textbf{Response}&  & $\textbf{Control}$ & $\mathbf{\%_{\mathrm{m}}}$ & $\textbf{Treated}$ & $\mathbf{\%_{\mathrm{n}}}$ & $\textbf{All}$ & $\mathbf{\%_{\mathrm{N}}}$ \\ 
  \hline
Officeholder & 0 & 16747 & 93.1 & 3058 & 93.2 & 19805 & 93.2 \\ 
   & 1 & 1234 & 6.9 & 222 & 6.8 & 1456 & 6.8 \\ 
   \hline
 & all & 17981 & 100.0 & 3280 & 100.0 & 21261 & 100.0 \\ 
  \end{tabularx}
  \begin{tabularx}{\linewidth}{l*{8}{Y}}
    \toprule
    \multicolumn{8}{l}{\textbf{Panel B: Combined sample}} \\
    \midrule
Officeholder & 0 & 16747 & 93.1 & 10813 & 93.9 & 27560 & 93.4 \\ 
   & 1 & 1234 & 6.9 & 705 & 6.1 & 1939 & 6.6 \\ 
   \hline
 & all & 17981 & 100.0 & 11518 & 100.0 & 29499 & 100.0 \\ 
    \bottomrule
  \end{tabularx}
 \footnotesize{Notes: Distribution of officeholders by treatment assignment for sample of 1805 lottery participants (Panel A) and combined sample of 1805 participants and 1807 fortunate drawers (Panel B). Both samples exclude women orphans, and pretreatment officeholders.}
\end{table}
\end{frame}

%Treatment effect on support for slavery
\begin{frame}
\frametitle{Results (cont.)}
\begin{table}[htb]
\caption{Support for slavery by treatment assignment.}   \label{outcomes-assembly}
  \begin{tabularx}{\linewidth}{l*{8}{Y}}
    \toprule
    \multicolumn{8}{l}{\textbf{Panel A: 1805 sample}} \\
    \midrule
 \textbf{Variable} & \textbf{Treatment} & $\mathbf{N}$ & \textbf{Min.} & $\mathbf{Mean}$ & \textbf{Max.} & $\mathbf{S.d.}$ & \textbf{\#NA}\\ 
  \hline
Support for slavery & 0 & 255 & 0 & 0.737 & 1 & 0.366 &  963 \\ 
   & 1 & 219 & 0 & 0.722 & 1 & 0.382 &  632 \\ 
   \hline
 & all & 474 & 0 & 0.730 & 1 & 0.373 & 1595 \\ 
   \hline
  \end{tabularx}
  \begin{tabularx}{\linewidth}{l*{8}{Y}}
    \toprule
    \multicolumn{8}{l}{\textbf{Panel B: Combined sample}} \\
    \midrule
Support for slavery & 0 & 401 & 0 & 0.735 & 1 & 0.371 & 1353 \\ 
   & 1 & 216 & 0 & 0.757 & 1 & 0.369 &  763 \\ 
   \hline
 & all & 617 & 0 & 0.743 & 1 & 0.370 & 2116 \\ 
    \bottomrule
  \end{tabularx} \\
\footnotesize{Distribution of the outcome variable, by treatment assignment, for 1805 participants (Panel A) or combined sample of 1805 participants and 1807 fortunate drawers (Panel B) who held office in the General Assembly before 1848 . `Support for slavery' is the mean of votes in favor of slavery.} 
\end{table}
\end{frame}

%Summary of results
\begin{frame}
\frametitle{Results (cont.)}
\begin{figure}[htbp]
\begin{center}
 %     \caption{Treatment effect estimates for each hypothesis test and sample used (horizontal lines represent 95\% confidence intervals)}
   \includegraphics[scale=0.37]{forest-plot.pdf} 
   \label{forest-plot} \\
   \end{center}
\end{figure}
\end{frame}

\section[Heterogeneous treatment effects]{}

\begin{frame}
\begin{figure}[htbp]
\begin{center}
      \caption{Heterogenous treatment effects for 1805 lottery participants.}
   \includegraphics[scale=0.35]{het-plot-oh.pdf} 
   \label{het-plot-oh}
   \end{center}
\end{figure}
\end{frame}

%Wealth densitiies
\begin{frame}
\begin{figure}[htbp]
\begin{center}
      \caption{Pre-- and posttreatment wealth densities for legislator--participants who voted on roll calls.}
   \includegraphics[scale=0.33]{wealth-plot.pdf} 
   \label{wealth-plot}
   \end{center}
\end{figure}
\end{frame}

\begin{frame}
\begin{figure}[htbp]
\begin{center}
      \caption{Heterogenous treatment effects for legislators who voted on roll calls.}
   \includegraphics[scale=0.35]{het-plot-assembly.pdf} 
   \label{het-plot-assembly}
   \end{center}
\end{figure}
\end{frame}

\section[Discussion]{}

\begin{frame}
\frametitle{Discussion}
\begin{itemize}
\item If property wealth influences political power, we should be able to find evidence in antebellum South
\item Officeholding: tight CIs on zero effect implies evidence ``in favor of'' null
\item Elite ideology: too much uncertainty to detect significant treatment effect
\begin{itemize}
\item Substantial heterogeneity in treatment effect according to pretreatment wealth
\end{itemize}
\end{itemize}
\end{frame}

\section[References]{}

\begin{frame}
\begin{singlespace}
\begin{tiny}
\bibliographystyle{plainnat}
\bibliography{../paper/refs}
\end{tiny}
\end{singlespace}
\itemize
\end{frame}

		
\end{document}

