\documentclass{beamer}
%%%%%%%%%%%%%%%%%%%%%%%%%%%%%%%%%%%%%%%%%%%%%%%%%%%%%%%%%%%%%%%

% Define packages
\usepackage{beamerthemesplit}
\usepackage{graphicx,amsfonts,psfrag,layout,subcaption,array,longtable,lscape,booktabs,dcolumn,natbib,amsmath,amssymb,amssymb,amsthm,setspace,epigraph,chronology,color, colortbl,caption}
\usepackage[]{graphicx}\usepackage[]{color}
\usepackage[page]{appendix}
\usepackage{hyperref, url} %For submission, uncheck and fix URLs ($$)
\usepackage[section]{placeins}
\usepackage[linewidth=1pt]{mdframed}

% Footnotes stick at the bottom
\usepackage[bottom]{footmisc}

% New footnote characters
\usepackage{footmisc}
\DefineFNsymbols{mySymbols}{{\ensuremath\dagger}{\ensuremath\ddagger}\S\P
   *{**}{\ensuremath{\dagger\dagger}}{\ensuremath{\ddagger\ddagger}}}
\setfnsymbol{mySymbols}

% New tabular environment
\usepackage{tabularx}
\newcolumntype{Y}{>{\raggedleft\arraybackslash}X}% raggedleft column X

% Define appendix 
\renewcommand*\appendixpagename{Appendix}
\renewcommand*\appendixtocname{Appendix}

% Position floats
\renewcommand{\textfraction}{0.05}
\renewcommand{\topfraction}{0.95}
\renewcommand{\bottomfraction}{0.95}
\renewcommand{\floatpagefraction}{0.35}
\setcounter{totalnumber}{5}

% Colors for highlighting tables
\definecolor{Gray}{gray}{0.9}

% Different font in captions
\newcommand{\captionfonts}{\scriptsize}

\makeatletter  % Allow the use of @ in command names
\long\def\@makecaption#1#2{%
  \vskip\abovecaptionskip
  \sbox\@tempboxa{{\captionfonts #1: #2}}%
  \ifdim \wd\@tempboxa >\hsize
    {\captionfonts #1: #2\par}
  \else
    \hbox to\hsize{\hfil\box\@tempboxa\hfil}%
  \fi
  \vskip\belowcaptionskip}
%\makeatother   % Cancel the effect of \makeatletter
 
% Number assumptions
\newtheorem*{assumption*}{\assumptionnumber}
\providecommand{\assumptionnumber}{}
\makeatletter
\newenvironment{assumption}[2]
 {%
  \renewcommand{\assumptionnumber}{Assumption #1}%
  \begin{assumption*}%
  \protected@edef\@currentlabel{#1}%
 }
 {%
  \end{assumption*}
 }
\makeatother

% Macros
\newcommand{\Adv}{{\mathbf{Adv}}}       
\newcommand{\prp}{{\mathrm{prp}}}                  % How to define new commands 
\newcommand{\calK}{{\cal K}}
\newcommand{\outputs}{{\Rightarrow}}                
\newcommand{\getsr}{{\:\stackrel{{\scriptscriptstyle\hspace{0.2em}\$}}{\leftarrow}\:}}
\newcommand{\andthen}{{\::\;\;}}    %  \: \; for thinspace, medspace, thickspace
\newcommand{\Rand}[1]{{\mathrm{Rand}[{#1}]}}       % A command with one argument
\newcommand{\Perm}[1]{{\mathrm{Perm}[{#1}]}}       
\newcommand{\Randd}[2]{{\mathrm{Rand}[{#1},{#2}]}} % and with two arguments
\newcommand{\E}{\mathrm{E}}
\newcommand{\Var}{\mathrm{Var}}
\newcommand{\Cov}{\mathrm{Cov}}
\DeclareMathOperator*{\plim}{plim}
\newcommand{\ind}{\mathbb{I}} % Indicator function
\newcommand{\pr}{\mathbb{P}} % Generic probability
\newcommand{\ex}{\mathbb{E}} % Generic expectation
\newcommand{\cov}{\mathrm{Cov}}
\newcommand\independent{\protect\mathpalette{\protect\independenT}{\perp}}
\def\independenT#1#2{\mathrel{\rlap{$#1#2$}\mkern2mu{#1#2}}}
\newcommand{\possessivecite}[1]{\citeauthor{#1}'s [\citeyear{#1}]} 

% Making a DAG
\usepackage{tkz-graph}  
\usetikzlibrary{shapes.geometric}
\tikzstyle{VertexStyle} = [shape            = ellipse,
                               minimum width    = 6ex,%
                               draw]
 \tikzstyle{EdgeStyle}   = [->,>=stealth']      


%color
\usecolortheme[RGB={50,200,50}]{structure} 

\usetheme[secheader]{Boadilla} 
\setbeamertemplate{items}[default] 
\setbeamercovered{transparent}
\setbeamertemplate{blocks}[rounded][shadow=true] 
\setbeamertemplate{navigation symbols}{} 
\mode<presentation>
\title[]{Estimating population average treatment effects from experiments with noncompliance}

\author[K. Ottoboni, J. Poulos]{Kellie Ottoboni \hspace{10mm} Jason Poulos}
%\institute[]{Stat 215B}
\date[10/15/15]{Oct. 15, 2015}
\begin{document}

\frame{\titlepage}

\section[Introduction]{}

\begin{frame}
\frametitle{Motivation}
\begin{itemize}
\item RCTs are the ``gold standard'' for estimating the causal effect of a treatment
\item External validity is an issue when RCT participants don't reflect the target population
\item Non-compliance to treatment assignment biases estimates of the sample average treatment effect (SATE) towards $0$
\end{itemize}
\end{frame}

\begin{frame}
\frametitle{Motivation (cont.)}
\begin{itemize}
\item Idea: reweight responses in the treatment group of RCT compliers to estimate population average treatment effect on the treated (PATT)
\item \cite{Hartman} develop a nonparametric reweighting method to extend SATE to PATT
\item We extend this method to the case of one-way crossover
\end{itemize}
\end{frame}

\section[Estimation]{}

\begin{frame}
\frametitle{Estimating treatment effects}
\begin{itemize}
\item Neyman-Rubin framework: each $i = \left\{1, ..., N \right\}$ participants have four potential outcomes, $Y_{ist}$ for $s = 0,1$ and $t = 0,1$
\begin{itemize}
\item S = study assignment: S=1 for RCT, S=0 for population/observational study
\item T = treatment assignment: T = 1 for treatment, T = 0 for control
\item D = treatment received
\end{itemize}
\item Other variables
\begin{itemize}
\item W = observed covariates
\item C = compliance to treatment
\item Y = response
\end{itemize}
\end{itemize}
\end{frame}


\begin{frame}
\frametitle{Estimating treatment effects (cont.)}
\begin{figure}[h]
\begin{tikzpicture}[scale=1.25] 
\SetGraphUnit{2} 
\Vertex{W}  \NOEA(W){S} \EA(S){T}  \EA(W){C} \SOEA(T){D} \SOEA(C){Y} \SOWE(D){Y}
\Edges(W, S, T) \Edges(W,T) \Edges(W, C) \Edges(T, D) \Edges(C, D) \Edges(W, Y) \Edges(D, Y)
\end{tikzpicture}
\caption{Causal diagram indicating the conditional independence assumptions needed to estimate the PATT.}\label{fig:DAG}
\end{figure}

\end{frame}


\begin{frame}
\frametitle{Estimating treatment effects (cont.)}
\begin{theorem}\label{thm1}
\fontsize{9pt}{7.2}\selectfont
Under assumptions \eqref{consistency} - \eqref{ER},

$$\tau_{\text{PATT}} = \ex_{01}\left[  \ex\left(Y_{11} \mid S=1, T=1, C=1, W\right)\right]
-\ex_{01}\left[  \ex\left(Y_{10} \mid S=1, T=0, C=1, W\right) \right] $$

where $\ex_{01}\left[\ex(\cdot \mid\dots, W)\right]$ denotes the expectation with respect to the distribution of $W$ in the treated individuals in the target population.  
\end{theorem}
\end{frame}


\section[Simulation]{}


\begin{frame}
\frametitle{Simulation Design}
\begin{itemize}
\item Generate a population of 30,000 with 3 observable covariates W
\item Set S, T, C, Y to be linear functions of W, with some Gaussian noise
\item Heterogeneous treatment effect: magnitude of effect depends on one of the covariates
\item Sample 5,000 ``randomizables'' for RCT and 5,000 ``observables'' for observational study. Enroll individuals according to S
\item Predict would-be compliers in the RCT control group using logistic regression
\item Estimate response curve in RCT compliers using a random forest
\item Use model to estimate potential outcomes in the observational study to estimate $\tau_{\text{PATT}}$
\end{itemize}
\end{frame}

%Simulation results plot
\begin{frame}
\begin{figure}[htbp]
%\caption{Balance in treatment assignment for 1805 sample.}
\centering
   \includegraphics[width=\linewidth]{../paper/mse_boxplots_B5.pdf} 
 %  \footnotesize{Note: $p$ values are calculated using a two--sided randomization test ($\mathcal{L}=1,000$ iterations) for weighted difference of means between treatment and control groups. Refer to footnotes in Table \ref{balance-nom} and Figure \ref{qq} for variable descriptions.}
\label{simulation-plot}
\end{figure}
\end{frame}



\section[Application]{}

\begin{frame}
\frametitle{Application: Oregon Health Insurance Experiment (OHIE)}
\begin{itemize}
\item In 2008, $\approx$ 90,000 uninsured low-income adults participated in a lottery to receive Medicaid benefits \citep{finkelstein2012}
\begin{itemize}
\item Selected participants won the opportunity for themselves and any household member to apply for Medicaid
\item 29,834 participants were selected by the lottery; remaining 45,008 served as controls 
\item Compliance measured by whether participant enrolled in Medicaid program during study period
\end{itemize}
\item Two health care use responses from mail survey (N = 23,741): emergency room (ER) and primary care visits in past 12 months
\end{itemize}
\end{frame}

\begin{frame}
\frametitle{Observational data}
\begin{itemize}
\item Data on the target population from National Health Interview Study (NHIS) \cite{NHIS} for 2009--2013
\item Restrict to respondents with income below 138\% of FPL and on Medicaid ($N=3,914$)
\item Covariates and responses match OHIE
\end{itemize}
\end{frame}

\begin{frame}
\frametitle{}
{\tiny
\begin{singlespace}
\begin{landscape}
\begin{longtable}{lllllll}
  & OHIE control &  & OHIE treated &  & NHIS treated&  \\ 
%    & control &  & treated &  &treated &   \\ 
  & $n=5476$ &  & $n=5193$ &  & $n=3382$ &  \\  
  \hline   
    \hline   
 \textbf{Covariate} &  $\mathbf{n}$ & $\mathbf{\%}$ & $\mathbf{n}$ & $\mathbf{\%}$ & $\mathbf{n}$ & $\mathbf{\%}$ \\ 
\hline
\textit{Sex:} &  & & &  &  & \\ 

\hspace{3mm} Female & 3148 & 57.5 & 2920 & 56.2 & 2380 & 70.4 \\ 
% &  & & &  &  & \\ 
\textit{Age:} &  & & &  &  & \\ 
\hspace{3mm}19-49 & 1636 & 29.9 & 1367 & 26.3 & 2429 & 71.8  \\ 

\hspace{3mm}50-64 & 3840 & 70.1 & 3826 & 73.7 & 953 & 28.2 \\ 
% &  & & &  &  & \\ 
\textit{Race:} &  & & &  &  & \\ 
\hspace{3mm}White & 4829 & 88.2 & 4393 & 84.6 & 1991 & 58.9  \\ 

\hspace{3mm}Black & 243 & 4.4 & 197 & 3.8 & 1050 & 31.1  \\ 

\hspace{3mm}Hispanic & 301 & 5.5 & 476 & 9.2 & 910 & 26.9  \\ 
% &  & & &  &  & \\ 
\textit{Health status:} &  & & &  &  & \\ 
\hspace{3mm}Diabetes & 581 & 10.6 & 539 & 10.4 & 452 & 13.4 \\ 

\hspace{3mm}Asthma & 1036 & 18.9 & 887 & 17.1 & 652 & 19.3  \\ 

\hspace{3mm}High blood pressure & 1670 & 30.5 & 1418 & 27.3 & 1143 & 33.8  \\ 
  
\hspace{3mm}Heart condition & 170 & 3.1 & 141 & 2.7 & 285 & 8.4 \\ 
% &  & & &  &  & \\ 
\textit{Education:} &  & & &  &  & \\  
\hspace{3mm}Less than high school  & 1056 & 19.3 & 950 & 18.3 & 1183 & 35.0  \\ 
  
\hspace{3mm}High school diploma or GED & 3081 & 56.3 & 2775 & 53.4 & 1076 & 31.8  \\ 

\hspace{3mm}Voc. training / 2-year degree & 969 & 17.7 & 1031 & 19.9 & 934 & 27.6 \\ 

\hspace{3mm}4-year college degree or more & 370 & 6.8 & 437 & 8.4 & 189 & 5.6 \\ 
% &  & & &  &  & \\ 
\textit{Income:} &  & & &  &  & \\ 
\hspace{3mm} $<\$10$k & 5476 & 100.0 & 3204 & 61.7 & 1452 & 42.9 \\

\hspace{3mm} \$10k-\$25k & 0 & 0.0 & 1616 & 31.1 & 1622 & 48.0 \\

\hspace{3mm} $>\$25$k & 0 & 0.0 & 373 & 7.2 & 308 & 9.1 \\
   \hline
\hline
 \textbf{Response} &   &  &  & &  &  \\ 
\hspace{3mm}Any ER visit &  1393 & 25.4 & 1301 & 25.1 & 881 & 26.1 \\ 
\hspace{3mm}Any outpatient visit & 3299 & 60.2 & 3081 & 59.3 & 2116 & 62.6 \\ 
\hline
\hline
%\caption{Pretreatment covariates and responses for OHIE and NHIS respondents on Medicaid.} 
\label{rct-nrt-compare}
\end{longtable}
\end{landscape}
\end{singlespace}
}
\end{frame}

\begin{frame}
\frametitle{Checking Assumptions} 
\begin{itemize}
\item Monotonicity is violated: two-way crossover occurred in OHIE
\begin{itemize}
\item 60\% of treated did not enroll in Medicaid 
\item 14\% of controls enrolled in Medicaid during the study period
\item Cross-over from control to treatment is low relative to other direction
\end{itemize}
\item Key assumption is strong ignorability: model of response given covariates is same for RCT and population
\begin{itemize}
%\item No way to check that we've included all possible confounders
\item We have included all possible confounders we have data on
\end{itemize}
\end{itemize}
\end{frame}

\begin{frame} 
\frametitle{Estimation Procedure}
\begin{enumerate}
\item Use ensemble method to predict complier status, given covariates, for RCT treated
\item Use ensemble fit to predict compliers among RCT controls 
\item For observed and predicted compliers, train random forests model to predict response using covariates and treatment as features
\item Using response model, estimate potential outcomes for population ``compliers'' on medicaid
\item $\tau_{\text{PATT}}$ is the difference in means between potential outcomes
%\item Estimate heterogeneous treatment effects by taking differences across response surfaces for each covariate group
\end{enumerate}
\end{frame}


\section[Results]{}

\begin{frame}
\begin{figure}[htbp]
\begin{center}
%       \caption{Any ER visit.}
   \includegraphics[scale=0.4]{../paper/any-visit-plot-horz.pdf} 
   \label{het-plot-av}
   \end{center}
\end{figure}
\end{frame}

\begin{frame}
\begin{figure}[htbp]
\begin{center}
%    \caption{Any primary care visit.}
   \includegraphics[scale=0.4]{../paper/any-out-plot-horz.pdf} 
   \label{het-plot-ao}
   \end{center}
\end{figure}
\end{frame}



\section[Conclusions]{}

\begin{frame}
\frametitle{Conclusions}
\begin{itemize}
\item Proposed estimator performs better than unadjusted estimator in simulations when compliance is low and can be predicted by observed covariates
\item Ensemble method for predicting compliance based on observed covariates has $77\%$ accuracy on the training set of OHIE treated
\item Substantial differences between sample and population estimates in terms of race, education, and health status subgroups
\end{itemize}
\end{frame}

\section[References]{}

\begin{frame}
\begin{singlespace}
\begin{tiny}
\bibliographystyle{plainnat}
\bibliography{../paper/refs}
\end{tiny}
\end{singlespace}
\itemize
\end{frame}

%APPENDIX 
\begin{frame}
\frametitle{Appendix: estimator assumptions}
\begin{assumption}{1}{}\label{consistency}
Consistency under parallel studies: for all $i$ and for $t=0, 1$,
$$Y_{i0t} = Y_{i1t}$$
\end{assumption}
\end{frame}

\begin{frame}
\frametitle{Appendix: estimator assumptions (cont.)}
\begin{assumption}{2}{}\label{si_treat}
Strong ignorability of sample assignment for treated:
\begin{equation*}
(Y_{01}, Y_{11}) \independent S \mid (W, T=1,C = 1), 0 < \pr(S=1 \mid W, T=1,C = 1) <1 
\end{equation*}
\end{assumption}

\begin{assumption}{3}{}\label{si_ctrl}
Strong ignorability of sample assignment for controls:
\begin{equation*}
(Y_{00}, Y_{10}) \independent S \mid (W, T=1,C = 1), 0 < \pr(S=1 \mid W, T=1,C = 1) <1 
\end{equation*}\end{assumption}

\noindent Potential outcomes are independent of sample assignment for individuals with the same covariates $W$ and assignment to treatment.
\end{frame}

\begin{frame}
\frametitle{Appendix: estimator assumptions (cont.)}
\begin{assumption}{4}{}\label{sutva}
Stable unit treatment value assumption (SUTVA):
\begin{equation*}
Y_{ist}^{L_i} = Y_{ist}^{L_j},  \forall i \neq j
\end{equation*}
where $L_j$ is the treatment and sample assignment vector for unit $j$. \end{assumption}
 
\begin{assumption}{5}{}\label{compl}
Conditional independence of compliance and assignment:
\begin{equation*}
C \independent T=1 \mid W, 0 < \pr(C = 1 \mid W) < 1
\end{equation*}
\end{assumption}
\end{frame}

\begin{frame}
\frametitle{Appendix: estimator assumptions (cont.)}
\begin{assumption}{6}{}\label{monotonicity}
Monotonicity: 
\begin{equation*}
T_i \geq D_i, \forall i
\end{equation*}
\end{assumption}
\noindent This assumption implies that there are no defiers and that crossover is only possible from treatment to control.
\begin{assumption}{7}{}\label{ER}
Exclusion restriction: For non-compliers
\begin{equation*}
Y_{11} = Y_{10}
\end{equation*}  
\end{assumption}
\noindent The treatment assignment affects the response only through the treatment received.  In particular, the treatment effect may only be non-zero for compliers.  
\end{frame}



		
\end{document}

